%!TEX program = xelatex
\documentclass[UTF8]{article}
\usepackage[UTF8]{ctex}
\usepackage[utf8]{inputenc}
\usepackage{hust_gp}
\addbibresource{reference.bib}
\pagestyle{fancy}
\cfoot{\thepage}
\geometry{a4paper,left=2.8cm,right=2.8cm,top=3cm,bottom=2.5cm}
% \geometry{b5paper,left=1.4cm,right=1.4cm,top=2.3cm,bottom=2cm}

% Rename your document info here
\title{标题}
\subtitle{副标题}
\author{作者}
\date{}

\begin{document}
  \pagenumbering{Roman}
    \begin{center}
    \oldincludegraphics[width=0.4\textwidth]{fig/logo.png}
  \end{center}
  \par \vspace{1em}
  {\xiaochuhao \textbf{本科毕业设计[论文]}}
  \vspace{8em}
  % 使用单倍行距(局部生效)
  {\linespread{1}\selectfont
    \maketitle
    \makesubtitle
  }
  \vspace{8em}
  \begin{center}
    \Large
    \begin{tabular}{p{2.8cm} >{\centering\arraybackslash}p{6cm}} % 两列居中对齐,没有垂直线或其他水平线
      \makebox[2.8cm][s]{\textbf{院系}} & 生命科学与技术学院 \\ \cline{2-2} % 第一行的第二列添加底部边框
      \makebox[2.8cm][s]{\textbf{专业班级}} & 生信基地 2201 班 \\ \cline{2-2} % 第二行的第二列添加底部边框
      \makebox[2.8cm][s]{\textbf{姓名}} & 张三 \\ \cline{2-2} % 第三行的第二列添加底部边框
      \makebox[2.8cm][s]{\textbf{学号}} & U2022114514 \\ \cline{2-2} % 第三行的第二列添加底部边框
      \makebox[2.8cm][s]{\textbf{指导教师}} & 李四教授、王五副教授 \\ \cline{2-2} % 第三行的第二列添加底部边框
    \end{tabular}
  \end{center}
  \vspace{5em}
  \begin{center}
    \sanhao
    \textbf{\today}
  \end{center}

  \thispagestyle{empty}
  \newpage
  \setcounter{page}{1}

  \newpage
  \thispagestyle{empty}
  \mbox{}

  \newpage
  \setcounter{page}{1}
  \begin{center}
  \CJKsf \xiaoerhao \textbf{学位论文原创性声明}
\end{center}

本人声明所呈交的学位论文是我个人在导师指导下进行的研究工作及取得的研究成果。尽我所知,除文中已经标明引用的内容外,本论文不包含任何其他个人或集体已经发表或撰写过的研究成果。对本文的研究做出贡献的个人和集体,均已在文中以明确方式标明。本人完全意识到本声明的法律结果由本人承担。

\vspace{2em}

\begin{flushright}
  作者签名:\hspace{6em} \hspace{2em}年\hspace{2em}月\hspace{2em}日
\end{flushright}

\vspace{3em}

\begin{center}
  \CJKsf \xiaoerhao \textbf{学位论文版权使用授权书}
\end{center}

本学位论文作者完全了解学校有关保留、使用学位论文的规定,即:学校有权保留并向国家有关部门或机构送交论文的复印件和电子版,允许论文被查阅和借阅。本人授权华中科技大学可以将本学位论文的全部或部分内容编入有关数据库进行检索,可以采用影印、缩印或扫描等复制手段保存和汇编本学位论文。

\begin{tabular}{ll}
本论文属于 & 保\hspace{1em}密$\square$,在\underline{\hspace{3em}}年解密后适用本授权书。\\
& 不保密$\square$。\\
\end{tabular}

\vspace{1em}

(请在以上方框内打"$\checkmark$")

\vspace{2em}

\begin{flushright}
  作者签名:\hspace{6em} \hspace{2em}年\hspace{2em}月\hspace{2em}日 \\
  导师签名:\hspace{6em} \hspace{2em}年\hspace{2em}月\hspace{2em}日
\end{flushright}


  \newpage
  \thispagestyle{empty}
  \mbox{}

  \newpage
  \setcounter{page}{1}
  \sectionstar{摘  要}

本文是华中科技大学学位论文的中文摘要示例。摘要内容应简明扼要地概括论文的研究目的、方法、结果和结论,通常不超过300字。摘要应突出论文的创新点和主要贡献,便于读者快速了解论文的核心内容。

{\CJKsf \sihao \textbf{关键词}}:华中科技大学;学位论文;中文摘要;示例

  \newpage
  \sectionstar{Abstract}

This is an example of an abstract for a thesis at Huazhong University of Science and Technology. The abstract should concisely summarize the research objectives, methods, results, and conclusions of the thesis, typically not exceeding 300 words. It should highlight the innovations and main contributions of the thesis, enabling readers to quickly grasp the core content.

{\sihao \textbf{Keywords}}: Huazhong University of Science and Technology; Thesis; Abstract; Example

  \newpage
  \tableofcontents

  \newpage
  \setcounter{page}{1}
  \pagenumbering{arabic}

  \section{绪论}

  这里是绪论这里是绪论这里是绪论这里是绪论这里是绪论这里是绪论这里是绪论这里是绪论这里是绪论这里是绪论这里是绪论这里是绪论这里是绪论这里是绪论这里是绪论这里是绪论这里是绪论这里是绪论这里是绪论这里是绪论这里是绪论这里是绪论这里是绪论这里是绪论这里是绪论这里是绪论这里是绪论这里是绪论这里是绪论这里是绪论这里是绪论这里是绪论这里是绪论这里是绪论这里是绪论这里是绪论这里是绪论这里是绪论这里是绪论这里是绪论这里是绪论这里是绪论这里是绪论这里是绪论这里是绪论这里是绪论这里是绪论这里是绪论这里是绪论这里是绪论这里是绪论这里是绪论这里是绪论这里是绪论这里是绪论这里是绪论这里是绪论这里是绪论这里是绪论这里是绪论这里是绪论这里是绪论这里是绪论这里是绪论这里是绪论这里是绪论这里是绪论这里是绪论这里是绪论这里是绪论这里是绪论这里是绪论这里是绪论这里是绪论这里是绪论这里是绪论这里是绪论这里是绪论这里是绪论这里是绪论这里是绪论这里是绪论这里是绪论这里是绪论这里是绪论这里是绪论这里是绪论这里是绪论这里是绪论这里是绪论这里是绪论这里是绪论这里是绪论这里是绪论这里是绪论这里是绪论这里是绪论这里是绪论这里是绪论这里是绪论这里是绪论这里是绪论这里是绪论这里是绪论这里是绪论这里是绪论这里是绪论这里是绪论这里是绪论这里是绪论这里是绪论这里是绪论这里是绪论这里是绪论

  \section{章节 1}

  这里是章节 1 的内容。This is section 1 content.这里是章节 1 的内容。This is section 1 content.这里是章节 1 的内容。This is section 1 content.这里是章节 1 的内容。This is section 1 content.这里是章节 1 的内容。This is section 1 content.这里是章节 1 的内容。This is section 1 content.这里是章节 1 的内容。This is section 1 content.这里是章节 1 的内容。This is section 1 content.这里是章节 1 的内容。This is section 1 content.这里是章节 1 的内容。This is section 1 content.\cite{test1}

  \section{章节 2}

  这里是章节 2 的内容。This is section 2 content.这里是章节 2 的内容。This is section 2 content.这里是章节 2 的内容。This is section 2 content.这里是章节 2 的内容。This is section 2 content.这里是章节 2 的内容。This is section 2 content.这里是章节 2 的内容。This is section 2 content.这里是章节 2 的内容。This is section 2 content.这里是章节 2 的内容。This is section 2 content.这里是章节 2 的内容。This is section 2 content.这里是章节 2 的内容。This is section 2 content.\cite{test2}

    \subsection{小节 2.1}

      这里是小节 2.1 的内容。This is subsection 2.1 content.这里是小节 2.1 的内容。This is subsection 2.1 content.这里是小节 2.1 的内容。This is subsection 2.1 content.这里是小节 2.1 的内容。This is subsection 2.1 content.这里是小节 2.1 的内容。This is subsection 2.1 content.这里是小节 2.1 的内容。$\sum_{i=1}^{n}n = \dfrac{n \times \left(1 + n\right)}{2}$This is subsection 2.1 content.这里是小节 2.1 的内容。This is subsection 2.1 content.这里是小节 2.1 的内容。This is subsection 2.1 content.这里是小节 2.1 的内容。This is subsection 2.1 content.这里是小节 2.1 的内容。This is subsection 2.1 content.\cite{test1}

      $$
        \begin{aligned}
          E & = mc^2 \\
          F & = ma \\
          V & = IR
        \end{aligned}
      $$

  \newpage
  \section{章节 3}

  \begin{figure}[htbp]
    \centering
    \mygraphics{fig/logo.png}
    \piccaption{示例图片}
    \label{fig:example}
  \end{figure}

  \begin{table}[htbp]
    \centering
    \tbcaption{示例表格}
    \begin{tabular}{ccc}
      \toprule
      项目 & 数值1 & 数值2 \\
      \midrule
      A & 10 & 20 \\
      B & 30 & 40 \\
      \bottomrule
    \end{tabular}
    \label{tab:example}
  \end{table}

  \begin{figure}[htbp]
    \piccaption{示例流程图}
    \label{fig:flowchart}
    \centering
    \begin{tikzpicture}[node distance=1.8cm, every node/.style={font=\small}]
      \node[process] (start) {开始};
      \node[process, right=of start] (p1) {步骤 1};
      \node[process, right=of p1] (p2) {步骤 2};
      \node[process, below=of p1] (p3) {步骤 3};
      \node[process, right=of p3] (end) {结束};

      \draw[arrow] (start) -- (p1);
      \draw[arrow] (p1) -- (p2);
      \draw[arrow] (p1) -- (p3);
      \draw[arrow] (p3) -- (end);
      \draw[twowayarrow] (p2.south) -- (p3.north);

      % Group box and annotation
      \node[edge, fit=(p1) (p2) (p3), label=above:流程组] {};
    \end{tikzpicture}
  \end{figure}

  \begin{minted}[linenos,fontsize=\small,breaklines]{python}
def fibonacci(n):
  if n <= 0:
    return []
  seq = [0, 1]
  while len(seq) < n:
    seq.append(seq[-1] + seq[-2])
  return seq[:n]
  \end{minted}

  \newpage
  \printbibliography[heading=bibintoc, title={参考文献}]
  % \bibliographystyle{unsrt}
  % \bibliography{reference}

\end{document}