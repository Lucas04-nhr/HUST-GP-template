  \begin{center}
    \oldincludegraphics[width=0.4\textwidth]{fig/logo.png}
    \par \vspace{1em}
    {\erhao \textbf{本科毕业设计(论文)开题报告}}
  \end{center}
  \vspace{8em}

  \maketitle
  
  \vspace{8em}
  \begin{center}
    \sanhao
    \begin{tabular}{p{2.8cm} >{\centering\arraybackslash}p{6cm}} % 两列居中对齐,没有垂直线或其他水平线
      \makebox[2.8cm][s]{\textbf{院系}} & 生命科学与技术学院 \\ \cline{2-2} % 第一行的第二列添加底部边框
      \makebox[2.8cm][s]{\textbf{专业班级}} & 生信基地 2201 班 \\ \cline{2-2} % 第二行的第二列添加底部边框
      \makebox[2.8cm][s]{\textbf{姓名}} & 张三 \\ \cline{2-2} % 第三行的第二列添加底部边框
      \makebox[2.8cm][s]{\textbf{学号}} & U2022114514 \\ \cline{2-2} % 第三行的第二列添加底部边框
      \makebox[2.8cm][s]{\textbf{指导教师}} & 李四教授、王五副教授 \\ \cline{2-2} % 第三行的第二列添加底部边框
    \end{tabular}
  \end{center}
  \vspace{5em}
  \begin{center}
    \sanhao
    \textbf{\today}
  \end{center}

  \thispagestyle{empty}
  \newpage

  \thispagestyle{empty}
  {\sihao
    \begingroup\renewcommand{\baselinestretch}{1}\selectfont
      \begin{center}
        \sanhao \textbf{开题报告填写要求}
      \end{center}

      \begin{enumerate}
        \renewcommand{\theenumi}{\chinese{enumi}}
        \renewcommand{\labelenumi}{\theenumi、}
        \item 开题报告主要内容:
          \begin{enumerate}
            \renewcommand{\theenumii}{\arabic{enumii}}
            \renewcommand{\labelenumii}{\theenumii.}
            \item 课题来源、目的、意义。
            \item 国内外研究现况及发展趋势。
            \item 预计达到的目标、关键理论和技术、主要研究内容、完成课题的方案及主要措施。
            \item 课题研究进度安排。
            \item 主要参考文献。
          \end{enumerate}
        \renewcommand{\theenumi}{\chinese{enumi}}
        \renewcommand{\labelenumi}{\theenumi、}
        \item 报告内容用小四号宋体字编辑,采用A4号纸双面打印,封面与封底采用浅蓝色封面纸(卡纸)打印。要求内容明确,语句通顺。
        \item 指导教师评语、教研室(系、所)或开题报告答辩小组审核意见用蓝、黑钢笔手写或小四号宋体字编辑,签名必须手写。
        \item 理、工、医类要求字数在3000字左右,文、管类要求字数在2000 字左右。
        \item 开题报告应在第八学期第二周之前完成。
      \end{enumerate}
    \endgroup
  }

  \thispagestyle{empty}
  \newpage