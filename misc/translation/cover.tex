  \begin{center}
    \oldincludegraphics[width=0.4\textwidth]{fig/logo.png}
  \end{center}
  \par \vspace{1em}
  {\erhao \textbf{本科毕业设计(论文)参考文献译文本}}
  \vspace{8em}


  
  \vspace{8em}
  \begin{center}
    \Large
    \begin{tabular}{p{2.8cm} >{\centering\arraybackslash}p{6cm}} % 两列居中对齐,没有垂直线或其他水平线
      \makebox[2.8cm][s]{\textbf{院系}} & 生命科学与技术学院 \\ \cline{2-2} % 第一行的第二列添加底部边框
      \makebox[2.8cm][s]{\textbf{专业班级}} & 生信基地 2201 班 \\ \cline{2-2} % 第二行的第二列添加底部边框
      \makebox[2.8cm][s]{\textbf{姓名}} & 张三 \\ \cline{2-2} % 第三行的第二列添加底部边框
      \makebox[2.8cm][s]{\textbf{学号}} & U2022114514 \\ \cline{2-2} % 第三行的第二列添加底部边框
      \makebox[2.8cm][s]{\textbf{指导教师}} & 李四教授、王五副教授 \\ \cline{2-2} % 第三行的第二列添加底部边框
    \end{tabular}
  \end{center}
  \vspace{5em}
  \begin{center}
    \sanhao
    \textbf{\today}
  \end{center}

  \thispagestyle{empty}
  \newpage

  \thispagestyle{empty}
  {\sihao
    \begingroup\renewcommand{\baselinestretch}{1}\selectfont
      \begin{center}
        \sanhao \textbf{译文要求}
      \end{center}

      \begin{enumerate}
        \renewcommand{\theenumi}{\chinese{enumi}}
        \renewcommand{\labelenumi}{\theenumi、}
        \item 译文内容须与课题(或专业内容)联系,并需在封面注明详细出处。
        \item 出处格式为 \\
        图书:作者.书名.版本(第×版).译者.出版地:出版者,出版年.起页$\sim$止页\\
        期刊:作者.文章名称.期刊名称,年号,卷号(期号):起页$\sim$止页
        \item 译文不少于5000汉字(或2万印刷符)
        \item 翻译内容用五号宋体字编辑,采用A4号纸双面打印,封面与封底采用浅蓝色封面纸(卡纸)打印。要求内容明确,语句通顺。
        \item 译文及其相应参考文献一起装订,顺序依次为封面、译文、参考文献原文。
        \item 翻译应在第七学期完成。
      \end{enumerate}

      \begin{center}
        \sanhao \textbf{译文评阅}
      \end{center}
      \lineB
      \vspace{1em}
      \noindent{\bf 导师评语}\par
      \noindent 应根据学校“译文要求”,对学生译文翻译的准确性、翻译数量以及译文的文字表述情况等做具体的评价后,再评分。\par
      \vspace{10em}
      \noindent 评分:\makebox[3cm]{\hrulefill}(百分制)\hfill 指导教师(签名):\makebox[3cm]{\hrulefill}
      \begin{flushright}
        年\hspace{3em}月\hspace{3em}日
      \end{flushright}
    \endgroup
  }

  \thispagestyle{empty}
  \newpage